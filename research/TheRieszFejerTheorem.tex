\documentclass{article}
\usepackage[english]{babel}
\usepackage{geometry,amsmath,amssymb,bbm}
\geometry{letterpaper}

%%%%%%%%%% Start TeXmacs macros
\newcommand{\mathd}{\mathrm{d}}
\newcommand{\tmop}[1]{\ensuremath{\operatorname{#1}}}
\newcommand{\tmtextit}[1]{{\itshape{#1}}}
%%%%%%%%%% End TeXmacs macros

\begin{document}

The Riesz-Fejer Theorem. {\cite[G.3p54]{li1}}

If $F (x) \geqslant 0 \forall x \in \mathbbm{R}$ is an entire exponential
function with
\begin{equation}
  \int_{- \infty}^{\infty} \frac{\log^+ (F (x))}{1 + x^2} \mathd x < \infty
\end{equation}
then there also exists an entire exponential function $f (x)$ without zeros in
the upper half-plane $\mathfrak{I} (x) > 0$ ($\mathfrak{I} \cong
\tmop{Im},$imaginary part) such that
\begin{equation}
  \text{} F (x) = f (x) \cdot \overline{f (x)} = | f (x) |^2 \forall x \in
  \mathbbm{R}
\end{equation}
\begin{thebibliography}{1}
  \bibitem[1]{li1}Paul Koosis.{\newblock} \tmtextit{The logarithmic integral
  I}.{\newblock} Cambridge University Press, 1988.{\newblock}
\end{thebibliography}

\

\end{document}
