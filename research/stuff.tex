\documentclass{article}
\usepackage{geometry,amssymb}
\usepackage[american]{babel}
\geometry{letterpaper}

%%%%%%%%%% Start TeXmacs macros
\newcommand{\mathd}{\mathrm{d}}
\newcommand{\tmem}[1]{{\em #1\/}}
%%%%%%%%%% End TeXmacs macros

\begin{document}

\section{Stuff:The Theory of Point Processes}

\subsection{Terminology and Whatnot}

Consider a (simple) $K$-dimensional {\tmem{multivariate point process}}
\begin{equation}
  F_t = \bigcup_{k = 1}^K F^k_t
\end{equation}
where $t_0^k < t_1^k < t_2^k < \ldots < t_i^k < \ldots$ and
\begin{equation}
  F^k_n = \bigcup_{n = 1}^{N^k_t} t^k_n
\end{equation}
and where $N^k_t$ denotes the {\tmem{counting process}} associated with the
$k$-th point process which is simply the number of events of P which have
occurred by time $t$. At time $t$, the most recent arrival time will be
denoted $t_{N^k_t}^k$. A process is said to be {\tmem{simple}} if no points
occur at the same time, that is, there are no zero-length durations. The
right-continuous(cadlag) counting
process{\cite[4.1.1.2]{hautsch2011econometrics}} can be represented as a sum
of Heaviside step functions $\theta ( t) = \left\{ \begin{array}{ll}
  0 & t < 0\\
  1 & t \geqslant 0
\end{array} \right.$
\begin{equation}
  N^k_t = \sum_{t_i^k \leqslant t}^{} \theta ( t - t_i^k)
\end{equation}
The counting function(process) jumps at the occurrence of each point and its
value is the number of points occurring up to the point in time of the jump,
inclusively. The left-continuous counting function does not include the time
of the most recent jump, it counts the number of events occurring
{\tmem{before}} t and is denoted by
\begin{equation}
  \breve{N}^k_t = \sum_{t_i^k < t} \theta ( t - t_i^k) = N_t^k - 1
\end{equation}
when $N$ is a simple counting process, that is, there are no events occuring
at the same time.

\subsection{The (Conditional) Intensity Function $\lambda^k ( t | F^{}_t)$}

The {\tmem{conditional intensity function}} gives the conditional probability
per unit time that an event of type $k$ occurs in the next instant given the
filtration $F_t = \bigcup_{k = 1}^K \{ t^k_i \}$ which is the set of all event
times, regardless of type/dimension, in the process preceeding the current
time $t$
\begin{equation}
  \lambda^k ( t | F^{}_t) = \lim_{\Delta t \rightarrow 0} \frac{\Pr ( N^k_{t +
  \Delta t} - N^k_t > 0 | F_t)}{\Delta t}
\end{equation}
For small values of $\Delta t$ we have
\begin{equation}
  \lambda^k ( t | F^{}_t) \Delta t = E ( N_{t + \Delta t}^k - N^k_t | F_t) + o
  ( \Delta t)
\end{equation}
such that the expectation
\begin{equation}
  E ( ( N^k_{t + \Delta t} - N^k_t) - \lambda^k ( t | F_t) \Delta t) = o (
  \Delta t)  \label{E}
\end{equation}
becomes uncorrelated with respect to the past of $F_t$ as $\Delta t
\rightarrow 0$. That is
\begin{equation}
  \begin{array}{l}
    E \left( \lim_{\Delta t \rightarrow 0} \left[ \sum_{j = 1}^{\frac{( u -
    s_{})}{\Delta t}} ( N^k_{u + j \Delta t} - N_{s + ( j - 1) \Delta t}^k) -
    \lambda^k ( s + j \Delta t | F_t) \Delta t \right] \right)\\
    = E \left( ( N_u^k - N^k_{s_{}}) - \lim_{\Delta t \rightarrow 0} \left[
    \sum_{j = 1}^{\frac{ ( u - s)}{\Delta t}} \lambda^k ( j \Delta t | F_t)
    \Delta t \right] \right)\\
    = E \left( ( N^k_u - N^k_{s_{}}) - \int_{s_{}}^u \lambda^k ( t | F_t)
    \mathd t \right)\\
    = E ( ( N^k_u - N^k_{s_{}}) - ( N^k_u - N^k_{s_{}}))\\
    = 0
  \end{array}
\end{equation}
since
\begin{equation}
  E \left( \int_s^u \lambda^k ( t | F_t) \mathd t \right) = N^k_u - N^k_s
\end{equation}

\subsection{The Compensator(aka Dual Predictble Projection) of a Point
Process}

The integrated intensity function is known as the {\tmem{compensator}}, or
more precisely, the {\tmem{$F_t$-compensator}} and will be denoted by
\begin{equation}
  \Lambda^k ( s_0, s_1) = \int_{s_0}^{s_1} \lambda^k ( t | F_t) \mathd t
  \label{compensator}
\end{equation}


Let $x_k = t_i^k - t_{i - 1}^k$ denote the time interval, or duration, between
the $i$-th and $( i - 1)$-th arrival times. The $F_t$-{\tmem{conditional
survivor function}} for the $k$-th process is given by
\begin{equation}
  S_k ( x_i^k) = P_k ( t_i^k > x_i^k | F_{t_{i - 1} + \tau})
\end{equation}
Let
\[ \tilde{\mathcal{E}}_i^k = \int_{t_{i - 1}}^{t_i} \lambda^k ( t | F_t)
   \mathd t = \Lambda^k ( t_{i - 1}, t_i)_{} \label{gr} \]
then provided the survivor function is absolutely continuous with respect to
Lebesgue measure(which is an assumption that needs to be verified, usually by
graphical tests) we have
\begin{equation}
  S^k ( x_i^k) = e^{- \int_{t_{i - 1}}^{t_i} \lambda^k ( t | F_t) \mathd t} =
  e^{- \tilde{\mathcal{E}}_i^k} \label{S}
\end{equation}
and $\tilde{\mathcal{E}}_{N ( t)}$ is an i.i.d. exponential random variable
with unit mean and variance. Since $E ( \tilde{\mathcal{E}}_{N ( t)}) = 1$ the
random variable
\begin{equation}
  \mathcal{E}_{N ( t)}^k = 1 - \tilde{\mathcal{E}}_{N ( t)}
\end{equation}
has zero mean and \ unit variance. Positive values of $\mathcal{E}_{N ( t)}$
indicate that the path of conditional intensity function $\lambda^k ( t |
F_t)$ under-predicted the number of events in the time interval and negative
values of $\mathcal{E}_{N ( t)}$ indicate that $\lambda^k ( t | F_t)$
over-predicted the number of events in the interval. \ In this way, (\ref{gr})
can be interpreted as a generalized residual. The {\tmem{backwards recurrence
time}} given by
\begin{equation}
  U_t^k = t - t_{N_t^k}
\end{equation}
increases linearly with jumps back to 0 at each new point.

\end{document}
